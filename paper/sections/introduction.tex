In recent years, artificial intelligence (AI) models have been gaining popularity in both personal use 
as well as in industry adoption. Among these models, none seem to have risen to such a high level of 
popular adoption as large language models (LLMs) which are AI models that can intake, understand, and generate
human-readable text \cite{brown2020languagemodelsfewshotlearners}. Usage of these has gained traction in a broad 
range of industries including 
research, healthcare, supply-chain, and software engineering 
\cite {liang2024mappingincreasingusellms} \cite{ZHANG2025102883} \cite{urlana2024llmsindustriallensdeciphering}. 
The most powerful LLMs are often built on 
an architectural framework called a transformer which calculates importance of different parts of the input and weighs 
the relationship between them in order to produce more coherent and relevant outputs.
In their seminal 2017 paper, Vaswani et al. introduced the 
transformer architecture and set the stage for the many iterations of LLMs that 
have come in the years since \cite{vaswani2023attentionneed}. This original paper focused on the core attention 
component of the transformer which calculates the interactions between different
input tokens and uses this contextual knowledge to more dynamically generate 
relevant output text. While a transformative architecture, the core attention mechanism does suffer from 
a few key drawbacks that have motivated the need for some important optimizations in the years since. 
Because it calculates interactions between every input token pair, the attention mechanism suffers from 
a quadratic time and memory complexity leading to some amnesia inducing context length limitations. Additionally, 
models immediately following the transformer architecture follow a single uniform application of the attention 
mechanism at their different layers. This lack of heterogenity and depth didn't allow the models to 
fully capture the nuanced and dynamic interactions between the input tokens.

Generative pre-trained transformers (GPTs) are a specific instantiation of LLMs that use the transformer architecture as a backbone. 
GPTs are engineered to be highly effective at generating relevant text and at performing 
conversationally oriented tasks by first pre-training them on a huge corpus of text and then fine tuning
them for their specific tasks- a process that was introduced by Radford et al. \cite{radford2018improving}. Following 
that initial introduction, GPTs utilizing magnitudes more parameters began to be explored- consistently
increasing their usefulness in conversational tasks. Specifically, post-training capacity of the 
models to learn and respond to instructions deliverd to them as inputs was detailed by Brown et al. in their 
2020 paper introducing GPT-3 \cite{brown2020languagemodelsfewshotlearners}. 
While making significant bounds in model performance and practical usefulness, these architectures still 
suffered from the same core limitations of the original transformer setup that they are built on top of. 

In the present work, we implement a from-scratch trained GPT architecture but we also explore some of the 
optimizations that have been introduced to address the aformentioned limitations of the transformer architecture. 
Two of the core optimizations that we utilize are flash-attention and multi-layer attention. Flash attention  addresses
the issue of quadratic memory complexity in the original attention mechanism- an issue that significantly slows down 
the computation as expensive GPU memory transfers abound with full attention scores being computed and scored. 
Flash attention focuses on making the attention mechanism input/output (IO) aware and splits the computations 
into tiles that can fit on a GPUs on-chip memory and eschew temporally costly IO operations \cite{dao2022flashattentionfastmemoryefficientexact}.
 In our setup, we use flash attention and see significant improvements in our model training time as well as inference.
